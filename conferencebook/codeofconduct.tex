\section{Code of Conduct}

The Midwest/Southeast Photosynthesis Conference is committed to providing a welcoming, inclusive, and harassment-free environment in all interactions regardless of race, age, ethnicity, national origin, language, gender, gender identity, sexual orientation, disability, physical appearance, political views, military service, health status or religion. We welcome the opportunity to bring photosynthesis research to all people regardless of their identity or background.

As a program that aims to share ideas and freedom of thought and expression, it is essential that the interaction between participants take place in an environment that recognizes the inherent worth of every person by being respectful of all. All participants strive to be empathetic, respectful, welcoming, friendly, and patient. We strive to be collaborative and use language that reflects our values.

The conference does not tolerate harassment in any form. Harassment is any form of behavior intended to exclude, intimidate or cause discomfort. Harassment includes, but is not limited to, the use of abusive or degrading language, intimidation, stalking, harassing photography or recording, inappropriate physical contact, and unwelcome sexual attention.

\subsection{Examples of unacceptable behavior}
All participants are committed to making participation in this community a harassment-free experience.

We will not accept harassment or other exclusionary behaviors or actions that are illegal, such as:

\begin{itemize}
	\item The use of sexualized language or imagery
	\item Excessive profanity (please avoid curse words; people differ greatly in their sensitivity to swearing)
	\item Posting sexually explicit or violent material
	\item Violent or intimidating threats or language directed against another person or group
	\item Inappropriate physical contact and/or unwelcome sexual attention or sexual comments
	\item Sexist, racist, or otherwise discriminatory jokes and language
	\item Trolling or insulting and derogatory comments
	\item Written or verbal comments which have the effect of excluding people on the basis of membership in a specific group, including level of experience, gender, gender identity and expression, sexual orientation, disability, neurotype, personal appearance, body size, race, ethnicity, age, religion, or nationality
	\item Public or private harassment
	\item Continuing to initiate interaction (such as photography, recording, messaging, or conversation) with someone after being asked to stop
	\item Sustained disruption of talks, events, or communications, such as heckling of a speaker
	\item Publishing (or threatening to post) other people’s personally identifying information (“doxing”), such as physical or electronic addresses, without explicit permission
	\item Other unethical or unprofessional conduct
	\item Advocating for, or encouraging, any of the above behaviors All participants are governed by local laws, including Turkey Run State Park rules and their organization’s code of conduct and policies.
\end{itemize}



\subsection{How to Submit a Report}
If you feel your safety is in jeopardy or the situation is an emergency, contact local law enforcement before making a report to the conference organizers. (In the U.S., dial 911.)

Anyone who experiences, observes or has knowledge of threatening behavior is expected to immediately report the incident to a member of the event organizing committee, state park staff, or a trusted friend or colleague. The Midwest/Southeast photosynthesis conference reserves the right to take appropriate action.

Take care of each other. Alert the organizers if you notice a dangerous situation, someone in distress, or violations of this code of conduct, even if they seem inconsequential. For possibly unintentional breaches of the code of conduct, you may want to respond to the person and point out this code of conduct (either in public or in private, whatever is most appropriate). If you would prefer not to do that, please report the issue to the conference chair or co-chair.

\subsubsection{What Happens Next}
All complaints will be reviewed and investigated and will result in a response that is deemed necessary and appropriate to the circumstances. All reports will be kept confidential, with the exception of cases where the organizers or state park staff determines the report should be shared with law enforcement. In those cases, the report will be shared with the proper legal authorities.

In some cases the conference organizers may determine that a public statement will need to be made. If that’s the case, the identities of all involved parties and reporters will remain confidential unless those individuals instruct us otherwise.